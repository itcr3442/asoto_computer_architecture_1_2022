\documentclass[conference, 14pt]{IEEEtran}

\usepackage[utf8]{inputenc}
\usepackage[spanish]{babel}

\usepackage[T1]{fontenc}
\usepackage{lmodern}

\usepackage{csquotes}
\usepackage{amsmath,amssymb,amsfonts}
\usepackage[per-mode=fraction, inter-unit-product=\cdot, quotient-mode=fraction]{siunitx}
\usepackage[shortlabels]{enumitem}

\usepackage[style=ieee]{biblatex}
\addbibresource{proyecto-i.bib}

\usepackage{graphicx}
\graphicspath{{./img}}

\usepackage{hyperref}
\hypersetup{colorlinks}

\begin{document}

\makeatletter
\newcommand{\linebreakand}{%
  \end{@IEEEauthorhalign}
  \hfill\mbox{}\par
  \mbox{}\hfill\begin{@IEEEauthorhalign}
}

\title{Proyecto I\\%
	\LARGE{Diseño e Implementación de un ASIP de desencriptación mediante RSA}}

\author{
	\IEEEauthorblockN{Alejandro Soto Chacón, 2019008164}
	\IEEEauthorblockA{CE4301: Arquitectura de Computadores I \\
		Instituto Tecnológico de Costa Rica}}

\maketitle

\pagestyle{plain}
\thispagestyle{plain}

\section{Diseño}

\subsection{Requerimientos}

\subsection{Opciones para ISA base}

\subsubsection{x86-64}

\subsubsection{AArch64}

\subsubsection{RV64I}

\subsection{Exponenciación modular}

\subsection{Técnicas de división de enteros}

\subsubsection{División por hardware}

\subsubsection{Inverso multiplicativo modular}

\subsubsection{Aritmética de punto fijo}

\subsection{Oportunidades de optimización en el caso concreto}

\subsubsection{Respecto a límites conocidos}

\subsubsection{Memoización de la relación ciphertext-plaintext}

\subsubsection{Vectorización del bucle principal}

\subsection{Propuestas de solución}

\subsubsection{RV64IMV}

\subsubsection{x86-64, AVX2}

\subsection{Comparación de propuestas}

\section{Implementación}

\subsection{Inicialización}

\subsubsection{Disposición en memoria y carriles SIMD}

\subsubsection{Precálculo de constantes para división rápida}

\subsection{Bucle principal}

\subsubsection{Tabla de búsqueda optimista}

\subsubsection{Exponenciación modular de LSB a MSB}

\subsubsection{Núcleo $\alpha := \alpha m \mod n$}

\subsection{Interfaz de usuario}

\subsubsection{Invocación}

\subsubsection{Visualización}

\subsection{Resultados}

\printbibliography[title={Referencias}]

\end{document}
